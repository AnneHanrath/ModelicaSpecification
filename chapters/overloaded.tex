\chapter{Overloaded Operators}\doublelabel{overloaded-operators}

A Modelica \textbf{operator record} can define the behavior for
operations such as constructing, adding, multiplying etc. This is done
using the specialized class \textbf{operator} (a restricted class
similar to \textbf{package}, see \ref{specialized-classes}) comprised of functions
implementing different variants of the operation for the record class in
which the respective \textbf{operator} definition resides. {[}\emph{The
overloading is defined in such a way that ambiguities are not allowed
and give an error. Furthermore, it is sufficient to define overloading
for scalars. Overloaded array operations are automatically deduced from
the overloaded scalar operations.}{]} The \textbf{operator} keyword is
followed by the name of the operation:

% This was something weird that could be seen as a table originally,
% that is somewhat awkward - but keeping it.
% 
% Changing to use itemize was attempted, but currently fails in LaTeXML
% https://github.com/brucemiller/LaTeXML/issues/1057
% (Note: It's pure coincide that this issue occured for this case.)
\begin{longtable}[c]{@{}|ll|@{}}
\hline\endhead
\multicolumn{2}{|l|}{Overloaded constructors, see \ref{overloaded-constructors}:}\\
& \lstinline!'constructor'!, \lstinline!'0'!\\
\multicolumn{2}{|l|}{Overloaded string conversions, see \ref{overloaded-string-conversions}:}\\
& \lstinline!'String'! \\
\multicolumn{2}{|l|}{Overloaded binary operations, see \ref{overloaded-binary-operations}:}\\
& \lstinline!'+'!, \lstinline!'-'! (subtraction), \lstinline!'*'!, \lstinline!'/'!, \lstinline!'^'!,\\
& \lstinline!'=='!, \lstinline!'<='!', \lstinline!'>'!, \lstinline!'<'!,
\lstinline!'>='!, \lstinline!'<='!, \lstinline!'and'!, \lstinline!'or'!\\
\multicolumn{2}{|l|}{Overloaded unary operations, see \ref{overloaded-unary-operations}:}\\
& \lstinline!'-'! (negation), \lstinline!'not'!\\
\hline
\end{longtable}

The functions defined in the operator-class must take at least one
component of the record class as input, except for the
constructor-functions which instead must return one component of the
record class. All of the functions shall return exactly one output.

The functions can be either called as defined in this section, or they
can be called directly using the hierarchical name. The operator or
operator function must be encapsulated; this allows direct calls of the
functions and prohibits the functions from using the elements of
operator record class.

The operator record may also contain additional functions, and
declarations of components of the record. It is not legal to extend from
an operator record, except as a short class definition modifying the
default attributes for the component elements directly inside the
operator record.

If an operator record was derived by a short class definition, the
overloaded operators of this operator record are the operators that are
defined in its base class, for subtyping see \ref{interface-or-type-relationships}.

The precedence and associativity of the overloaded operators is
identical to the one defined in Table 3.1 in \ref{operator-precedence-and-associativity}.

{[}\emph{Note, the operator overloading as defined in this section is
only a short hand notation for function calls.}{]}

\section{Matching Function}\doublelabel{matching-function}

All functions defined inside the \textbf{operator} class must return one
output (based on the restriction above), and may include functions with
optional arguments, i.e. functions of the form

\begin{lstlisting}[language=modelica,mathescape=true]
function f
  input $A_1$ $u_1$;
  ...
  input $A_{m}$ $u_{m}$ := $A_{m}$;
  ...
  input $A_{n}$ $u_{n}$;
  output B y;
algorithm
  ...
end f;
\end{lstlisting}
The vector P indicates whether argument m of f has a default value
(\textbf{true} for default value, \textbf{false} otherwise). A call
f($A_1$, $a_{2}$,\ldots{}, $a_{k}$,
$b_{1}$ = $w_{1}$ ,\ldots{}, $b_{p}$=
$w_{p}$) with distinct names $b_{j}$ is a valid
match for the function f, provided (treating Integer and Real as the
same type)

\begin{itemize}
\item
  $A_{i}$ = typeOf($A_{i}$) for 1 $\le$ i $\le$ k,
\item
  the names $b_{j}$ = $u_{Qj}$, Qj \textgreater{}
  k, $A_{Qj}$ =typeOf($w_{i}$) for 1 $\le$ j $\le$ p, and
\item
  if the union of \{i: 1 $\le$ i $\le$ k \}, \{Qj: 1 $\le$ j $\le$ p\}, and \{m:
  $P_{m}$ \textbf{true} and 1 $\le$ m $\le$ n \} is the set \{i: 1 $\le$
  i $\le$ n\}.
\end{itemize}

{[}\emph{This corresponds to the normal treatment of function calls with
named arguments, requiring that all inputs have some value given by a
positional argument, named argument, or a default value (and that
positional and named arguments do not overlap). Note, that this only
defines a valid call, but does not explicitly define the set of
domains.}{]}

\section{Overloaded Constructors}\doublelabel{overloaded-constructors}

Let C denote an operator record class and consider an expression
C($A_1$, $a_{2}$,\ldots{}, $a_{k}$,
$b_{1}$= $w_{1}$ ,\ldots{}, $b_{p}$=
$w_{p}$).

\begin{enumerate}
\item
  If there exists a {unique} function f in C.'constructor' such that
  ($A_1$, $a_{2}$,\ldots{}, $a_{k}$,
  $b_{1}$= $w_{1}$ ,\ldots{}, $b_{p}$=
  $w_{p}$) is a valid match for the function f, then
  C($A_1$, $a_{2}$,\ldots{}, $a_{k}$,
  $b_{1}$= $w_{1}$ ,\ldots{}, $b_{p}$=
  $w_{p}$) is resolved to\\
  C.'constructor'.f($A_1$, $a_{2}$,\ldots{},
  $a_{k}$, $b_{1}$= $w_{1}$ ,\ldots{},
  $b_{p}$= $w_{p}$).
\item
  If there is no operator C. 'constructor' the automatically generated
  record constructor is called.
\item
  Otherwise the expression is erroneous.
\end{enumerate}

Restrictions:

\begin{itemize}
\item
  The operator C.'constructor' shall only contain functions that declare
  one output component, which shall be of the operator record class C.
\item
  For an operator recordclass there shall not exist any potential call
  that lead to multiple matches in (1) above. {[}\emph{How to verify
  this is not specified.}{]}
\item
  For a pair of operator record classes C and D and components c and d
  of these classes both of C.'constructor' (d) and D.'constructor' (c)
  shall not both be legal {[}\emph{, so one of the two definitions must
  be removed}{]}\emph{.}
\end{itemize}

{[}\emph{By the last restriction the following problem for binary
operators is avoided: }

\emph{Assume there are two operator record classes C and D that both
have a constructor from Real. If we want to extend c+c and d+d to
support mixed operations, one variant would be to define c+d and d+c;
but then c+2 becomes ambiguous (since it is not clear which instance
should be converted to). Without mixed operations expressions such as
c+d are only ambiguous if both conversion from C to D and back from D to
C are both available, and this possibility is not allowed by the
restriction above.}{]}

Additionally there is an operator `0' defining the zero-value which can
also be used to construct an element. The operator `0' for an operator
record C can contain only one function, having zero inputs and one
output -- of class C (the called function is therefore unambiguous). It
should return the identity element of addition, and is used for
generating flow-equations for connect-equations and zero elements for
matrix-multiplication.

\section{Overloaded String Conversions}\doublelabel{overloaded-string-conversions}

Consider an expression String($A_1$,
$a_{2}$,\ldots{}, $a_{k}$, $b_{1}$=
$w_{1}$ ,\ldots{}, $b_{p}$= $w_{p}$), $k\ge 1$ where $A_1$ is an element of class A.

\begin{enumerate}
\item
  If A is a predefined {type}, i.e., Boolean, Integer, Real, String or
  an enumeration, or a type derived from them, then the corresponding
  built-in operation is performed.
\item
  If A is an operator {record class} and there exists a unique function
  f in A.'String' such that ($A_1$,
  $a_{2}$,\ldots{}, $a_{k}$, $b_{1}$=
  $w_{1}$ ,\ldots{}, $b_{p}$= $w_{p}$)
  is a valid match for f, then String($A_1$,
  $a_{2}$,\ldots{}, $a_{k}$, $b_{1}$=
  $w_{1}$ ,\ldots{}, $b_{p}$= $w_{p}$)
  is evaluated to\\
  A.'String'.f($A_1$, $a_{2}$,\ldots{},
  $a_{k}$, $b_{1}$= $w_{1}$ ,\ldots{},
  $b_{p}$= $w_{p}$).
\item
  Otherwise the expression is erroneous.
\end{enumerate}

Restrictions:

\begin{itemize}
\item
  The operator A.'String' shall only contain functions that declare one
  output component, which shall be of the String type, and the first
  input argument shall be of the operator record class A.
\item
  For an operator record class there shall not exist any call that lead
  to multiple matches in (2) above. {[}\emph{How to verify this is not
  specified.}{]}
\end{itemize}

\section{Overloaded Binary Operations}\doublelabel{overloaded-binary-operations}

Let \emph{op} denote a binary operator and consider an expression
\emph{a op b} where \emph{a} is an instance or array of instances of
class \emph{A} and \emph{b} is an instance or array of instances of
class \emph{B}.

\begin{enumerate}
\item
  If \emph{A} and \emph{B} are predefined types of such, then the
  corresponding built-in operation is performed.
\item
  Otherwise, if there exists {exactly one} function \emph{f} in the
  union of \emph{A}.\emph{op} and \emph{B.op} such that
  f(\emph{a},\emph{b}) is a valid match for the function \emph{f} , then
  \emph{a op b} is evaluated using this function. It is an error, if
  multiple functions match. If A is not an operator record class, A.op
  is seen as the empty set, and similarly for B. {[}\emph{Having a union
  of the operators ensures that if A and B are the same, each function
  only appears once.}{]}
\item
  Otherwise, consider the set given by \emph{f} in \emph{A}.\emph{op}
  and an operator record class \emph{C} (different from B) with a
  constructor, g, such that C.'constructor'.g(b) is a valid match, and
  f(a, C.'constructor'.g(b)) is a valid match; and another set given by
  \emph{f} in \emph{B}.\emph{op} and an operator record class \emph{D}
  (different from A) with a constructor, h, such that
  D.'constructor'.h(a) is a valid match and f(D.'constructor'.h(a), b)
  is a valid match. If the sum of the sizes of these sets is one this
  gives the unique match. If the sum of the sizes is larger than one it
  is an error.
  
  {[}\emph{Informally, this means: If there is no direct match of ``a op
  b'', then it is tried to find a direct match by automatic type casts
  of ``a'' or ``b'', by converting either ``a'' or ``b'' to the needed
  type using an appropriate constructor function from one of the
  operator record classes used as arguments of the overloaded ``op''
  functions. Example using the Complex-definition below:}
  
\begin{lstlisting}[language=modelica]
   Real a;
  Complex b;
  Complex c = a*b; // interpreted as:
  // Complex.'*'.multiply(Complex.'constructor'.fromReal(a),b);
\end{lstlisting}
{]}
\item
  Otherwise, if a or b is an {array expression}, then the expression is
  conceptually evaluated according to the rules of \ref{scalar-vector-matrix-and-array-operator-functions} with the
  following exceptions concerning \ref{matrix-and-vector-multiplication-of-numeric-arrays}:

  \begin{enumerate}
  \def\labelenumii{(\alph{enumii})}
  \item
    vector*vector should be left undefined {[}\emph{as the scalar
    product of} \emph{Table 10.16 does not generalize to the expected
    linear and conjugate linear scalar product of complex
    numbers}{]}\emph{.}
  \item
    vector*matrix should be left undefined {[}\emph{as the corresponding
    definition of} \emph{Table 10.16 does not generalize to complex
    numbers in the expected way}{]}.
  \item
    If the inner dimension for matrix*vector or matrix*matrix is zero,
    this uses the overloaded '0' operator of the result array element
    type. If the operator '0' is not defined for that class it is an
    error if the inner dimension is zero.
  \end{enumerate}
{[}\emph{For array multiplication it is assumed that the scalar elements
form a non-commutative ring that does not necessarily have a
multiplicative identity.}{]}
\item
  Otherwise the expression is erroneous.
\end{enumerate}

For an element-wise operator, a .op b, items 1, 4, and 5 are used; e.g.
the operator .+ will always be defined in terms of '+'.

Restrictions:

\begin{itemize}
\item
  A function is allowed for a binary operator if and only if it has at
  least two inputs; at least one of which is of the operator record
  class, and the first two inputs shall not have default values, and all
  inputs after the first two must have default values.
\item
  For an operator record class there shall not exist {any}
  {[}\emph{potential}{]} call that lead to multiple matches in (2)
  above\emph{.}
\end{itemize}

\section{Overloaded Unary Operations}\doublelabel{overloaded-unary-operations}

Let \emph{op} denote a unary operator and consider an expression
\emph{op a} where \emph{a} is an instance or array of instances of class
\emph{A}. Then \emph{op a} is evaluated in the following way.

\begin{enumerate}
\item
  If \emph{A} is a {predefined type}, then the corresponding built-in
  operation is performed.
\item
  If \emph{A} is an {operator record class} and there exists a {unique}
  function \emph{f} in \emph{A.op} such that \emph{A.op.f(a)} is a valid
  match, then \emph{op a} is evaluated to \emph{A.op.f(a)}. It is an
  error, if there are multiple valid matches.
\item
  Otherwise, if \emph{a} is an {array expression}, then the expression
  is conceptually evaluated according to the rules of \ref{scalar-vector-matrix-and-array-operator-functions}.
\item
  Otherwise the expression is erroneous.
\end{enumerate}

Restrictions:

\begin{itemize}
\item
  A function is allowed for a unary operator if and only if it has least
  one input; and the first input is of the record type (or suitable
  arrays of such) and does not have a default value, and all inputs
  after the first one must have default values.
\item
  For an operator record class there shall not exist {any}
  {[}\emph{potential}{]} call that lead to multiple matches in (2)
  above.
\item
  A binary and/or unary operator-class may only contain functions that
  are allowed for this binary and/or unary operator-class; and in case
  of `-` it is the union of these sets, since it may define both a unary
  (negation) and binary (subtraction) operator.
\end{itemize}

\section{Example of Overloading for Complex Numbers}\doublelabel{example-of-overloading-for-complex-numbers}

{[}\emph{The rules in the previous subsections are demonstrated at hand
of a record class to work conveniently with complex numbers:}

\begin{lstlisting}[language=modelica,escapechar=!]
operator record Complex "Record defining a Complex number"
  Real re "Real part of complex number";
  Real im "Imaginary part of complex number";
  encapsulated operator 'constructor'
    import Complex;
    function fromReal
      input Real re;
      input Real im := 0;
      output Complex result(re=re, im=im);
    algorithm
      annotation(Inline=true);
    end fromReal;
  end 'constructor';

  encapsulated operator function '+' // short hand notation, see !\ref{specialized-classes}!
    import Complex;
    input Complex c1;
    input Complex c2;
    output Complex result "= c1 + c2";
  algorithm 
    result := Complex(c1.re + c2.re, c1.im + c2.im);
   annotation(Inline=true);
  end '+';

  encapsulated operator '-'
    import Complex;
    function negate
      input Complex c;
      output Complex result "= - c";
    algorithm 
      result := Complex(-c.re, -c.im);
      annotation(Inline=true);
    end negate;

    function subtract
      input Complex c1;
      input Complex c2;
      output Complex result "= c1 -- c2";
    algorithm 
      result := Complex(c1.re - c2.re, c1.im - c2.im);
      annotation(Inline=true);
    end subtract;
  end '-';

  encapsulated operator function '*'
    import Complex;
    input Complex c1;
    input Complex c2;
    output Complex result "= c1 * c2";
  algorithm 
    result := Complex(c1.re*c2.re - c1.im*c2.im, c1.re*c2.im + c1.im*c2.re);
    annotation(Inline=true);
  end '*';

  encapsulated operator function '/'
    import Complex; input Complex c1;
    input Complex c2;
    output Complex result "= c1 / c2";
  algorithm 
    result := Complex(( c1.re*c2.re + c1.im*c2.im)/(c2.re^2 +
    c2.im^2),
    (-c1.re*c2.im + c1.im*c2.re)/(c2.re^2 + c2.im^2));
   annotation(Inline=true);
  end '/';

  encapsulated operator function '=='
    import Complex;
    input Complex c1;
    input Complex c2;
    output Boolean result "= c1 == c2";
  algorithm 
    result := c1.re == c2.re and c1.im == c2.im;
   annotation(Inline=true);
 end '==';

  encapsulated operator function 'String'
    import Complex;
    input Complex c;
    input String name := "j" "Name of variable representing sqrt(-1) in the string";
    input Integer significantDigits=6 "Number of significant digits to be shown";
    output String s;
  algorithm 
    s := String(c.re, significantDigits=significantDigits);
    if c.im <> 0 then
      s := if  c.im > 0 then s + " + "
   else s + " - ";
      s := s + String(abs(c.im), significantDigits=significantDigits) + name;
   end if;
  end 'String';

  encapsulated function j
    import Complex;
    output Complex c;
  algorithm 
    c := Complex(0,1);
    annotation(Inline=true);
  end j;

  encapsulated operator function '0'
    import Complex;
    output Complex c;
  algorithm 
    c := Complex(0,0);
    annotation(Inline=true);
  end '0';
end Complex;

function eigenValues
  input Real A [:,:];
  output Complex ev[size(A, 1)];
  protected 
  Integer nx=size(A, 1);
  Real eval[nx,2];
  Integer i;
algorithm 
  eval := Modelica.Math.Matrices.eigenValues(A);
  for i in 1:nx loop
    ev[i] := Complex(eval[i, 1], eval[i, 2]);
  end for;
end eigenValues;

// Usage of Complex number above:
  Complex j = Complex.j();
  Complex c1 = 2 + 3*j;
  Complex c2 = 3 + 4*j;
  Complex c3 = c1 + c2;
  Complex c4[:] = eigenValues([1,2; -3,4]);
algorithm
  Modelica.Utilities.Streams.print("c4 = " + String(c4));
  // results in output:
  // c4 = {2.5 + 1.93649j, 2.5 - 1.93649j}
\end{lstlisting}
\emph{How overloaded operators can be symbolically processed. Example:}

\begin{lstlisting}[language=modelica]
  Real a;
  Complex b;
  Complex c = a + b;
\end{lstlisting}
\emph{Due to inlining of functions, the equation for ``c'' is
transformed to:}
\begin{lstlisting}[language=modelica]
c = Complex.'+'.add(Complex.'constructor'.fromReal(a), b);
  = Complex.'+'.add(Complex(re=a,im=0), b)
  = Complex(re=a+b.re, im=b.im);
\end{lstlisting}
\emph{or}
\begin{lstlisting}[language=modelica]
  c.re = a + b.re;
  c.im = b.im;
\end{lstlisting}
\emph{These equations can be symbolically processed as other equations.}

\emph{Complex can be used in a connector:}
\begin{lstlisting}[language=modelica]
  operator record ComplexVoltage = Complex(re(unit="V"),im(unit="V"));
  operator record ComplexCurrent = Complex(re(unit="A"),im(unit="A"));

  connector ComplexPin
    ComplexVoltage v;
    flow ComplexCurrent i;
  end ComplexPin;
  
  ComplexPin p1,p2,p3;
equation
  connect(p1,p2);
  connect(p1,p3);
\end{lstlisting}
\emph{The two connect equations result in the following connection
equations:}
\begin{lstlisting}[language=modelica]
  p1.v = p2.v;
  p1.v = p3.v;
  p1.i + p2.i + p3.i = Complex.'0'();
  // Complex.'+'(p1.i, Complex.'+'(p2.i, p3.i)) = Complex.'0'();
\end{lstlisting}
\emph{The restrictions on extends are intended to avoid combining two
variants inheriting from the same operator record, but with possibly
different operations; thus ComplexVoltage and ComplexCurrent still use
the operations from Complex. The restriction that it is not legal to
extend from any of its enclosing scopes implies that:}

\begin{lstlisting}[language=modelica]
package A
  extends Icon; //Ok.
  operator record B ... end B;
  end A;

  package A2
    extends A(...); // Not legal
  end A2;
  package A3=A(...); // Not legal
\end{lstlisting}
{]}
