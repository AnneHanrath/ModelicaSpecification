\chapter{Packages}\doublelabel{packages}

\section{Package as Specialized Class}\doublelabel{package-as-specialized-class}

The package concept is a specialized class (\ref{specialized-classes}), using the
keyword package.

\section{Motivation and Usage of Packages}\doublelabel{motivation-and-usage-of-packages}

{[}\emph{Packages in Modelica may contain definitions of constants and
classes including all kinds of specialized classes, functions, and
subpackages. By the term subpackage we mean that the package is declared
inside another package, no inheritance relationship is implied.
Parameters and variables cannot be declared in a package. The
definitions in a package should typically be related in some way, which
is the main reason they are placed in a particular package. Packages are
useful for a number of reasons:}

\begin{itemize}
\item
  Definitions that are related to some particular topic are typically
  grouped into a package. This makes those definitions easier to find
  and the code more understandable.
\item
  \emph{Packages provide encapsulation and coarse-grained structuring
  that reduces the complexity of large systems. An important example is
  the use of packages for construction of (hierarchical) class
  libraries.}
\item
  \emph{Name conflicts between definitions in different packages are
  eliminated since the package name is implicitly prefixed to names of
  definitions declared in a package.}
\item
  \emph{Information hiding and encapsulation can be supported to some
  extent by declaring} protected \emph{classes, types, and other
  definitions that are available only inside the package and therefore
  inaccessible to outside code.}
\item
  \emph{Modelica defines a method for locating a package by providing a
  standard mapping of package names to storage places, typically file or
  directory locations in the file system. }
\end{itemize}

{]}

\subsection{Importing Definitions from a Package}\doublelabel{importing-definitions-from-a-package}

The import-clause makes public classes and other public definitions
declared in some package available for use by shorter names in a class
or a package. It is the only way of referring to definitions declared in
some other package for use inside an encapsulated package or class.

{[}\emph{Import-clauses in a package or class fill the following two
needs:}

\begin{itemize}
\item
  Making definitions from other packages available for use (by shorter
  names) in a package or class.
\item
  \emph{Explicit declaration of usage dependences on other packages.}
\end{itemize}

{]}

An import-clause can occur in one of the following five syntactic forms:

\textbf{import} \emph{packagename}; (qualified import)

\textbf{import} {[}\emph{packagename}.{]}\emph{definitionname}; (single
definition import)

\textbf{import}
{[}\emph{packagename}.{]}\{\emph{def1,def2,\ldots{}defN}\}; (multiple
definition import)

\textbf{import} \emph{packagename}.*; (unqualified import)

\textbf{import} \emph{shortpackagename} = \emph{packagename}; (renaming
import)

\textbf{import} \emph{shortpackagename} =
{[}\emph{packagename.{]}definitionname}; (renaming single def. import)

Here \emph{packagename} is the fully qualified name of the imported
package including possible dot notation and \emph{definitionname} is the
name of an element in a package. The multiple definition import is
equivalent to multiple single definition imports with corresponding
packagename and definition names.

\subsubsection{Lookup of Imported Names}\doublelabel{lookup-of-imported-names}

This section only defines how the imported name is looked up in the
import clause. For lookup in general -- including how import clauses are
used, see \ref{static-name-lookup}.

Lookup of the name of an imported package or class, e.g. A.B.C in the
clauses import A.B.C; import D=A.B.C; import A.B.C.*, deviates from the
normal lexical lookup by starting the lexical lookup of the first part
of the name at the top-level.

Qualified import clauses may only refer to packages or elements of
packages, i.e., in import A.B.C; or import D=A.B.C;, A.B must be a
package. Unqualified import clauses may only import from packages, i.e.,
in import A.B.*;, A.B must be a package. {[}\emph{Note: in} import A;
\emph{the class} A \emph{can be any class which is an element of the
unnamed top-level package}{]}

{[}\emph{For example, if the package} ComplexNumbers \emph{would have
been declared as a subpackage inside the package} Modelica.Math\emph{,
its fully qualified name would be} Modelica.Math.ComplexNumbers\emph{.
Definitionname is the simple name without dot notation of a single
definition that is imported. A shortpackagename is a simple name without
dot notation that can be used to refer to the package after import
instead of the presumably much longer packagename.}

\emph{The forms of import are exemplified below assuming that we want to
access the addition operation of the hypothetical package}
Modelica.Math.ComplexNumbers:

\begin{lstlisting}[language=modelica]
  import Modelica.Math.ComplexNumbers; // Accessed by ComplexNumbers.Add
  import Modelica.Math.ComplexNumbers.Add; // Accessed by Add
  import Modelica.Math.ComplexNumbers.{Add,Sub}; // Accessed by Add and Sub
  import Modelica.Math.ComplexNumbers.*; // Accessed by Add
  import Co = Modelica.Math.ComplexNumbers; // Accessed by Co.Add
\end{lstlisting}
{]}

\subsubsection{Summary of Rules for Import Clauses}\doublelabel{summary-of-rules-for-import-clauses}

The following rules apply to import-clauses:

\begin{itemize}
\item
  Import-clauses are \emph{not} inherited.
\item
  Import-clauses are not named elements of a class or package. This
  means that import-clauses cannot be changed by modifiers or
  redeclarations.
\item
  The \emph{order} of import-clauses does not matter.
\item
  One can only import \emph{from} packages, not from other kinds of
  classes. Both packages and classes can be imported \emph{into} i.e.,
  they may contain import-clauses.
\item
  An imported package or definition should always be referred to by its
  fully qualified name in the import-clause.
\item
  Multiple qualified import-clauses may not have the same import name.
\end{itemize}

\subsection{Mapping Package/Class Structures to a Hierarchical File System}\doublelabel{mapping-package-class-structures-to-a-hierarchical-file-system}

Packages/classes may be represented in the hierarchical structure of the
operating system {[}\emph{the file system}{]}. For classes with version
information see also \ref{mapping-of-versions-to-file-system}. The nature of such an external
entity falls into one of the following two groups:

\begin{itemize}
\item
  Directory in the file system.
\end{itemize}

\begin{itemize}
\item
  File in the file system.
\end{itemize}

Each Modelica file in the file-system is stored in UTF-8 format (defined
by The Unicode Consortium; http://www.unicode.org) and may start with
the UTF-8 encoded byte order mark (0xef 0xbb 0xbf); this is treated as
white-space in the grammar. \emph{{[}Tools may also store classes in
data-base systems, but that is not standardized.{]}}

\subsubsection{Mapping a Package/Class Hierarchy into a Directory Hierarchy (Structured Entity)}\doublelabel{mapping-a-package-class-hierarchy-into-a-directory-hierarchy-structured-entity}

A directory shall contain a node, the file package.mo. The node shall contain a stored-definition that defines a class {[}A{]} with a name
matching the name of the structured entity. {[}\emph{The node typically
contains documentation and graphical information for a package, but may
also contain additional elements of the class} A.{]}

A directory may also contain one or more sub-entities (directories or
files). The sub-entities are mapped as elements of the class defined by
their enclosing structured entity. {[}\emph{For example, if directory} A
\emph{contains the three files} package.mo, B.mo \emph{and} C.mo
\emph{the classes defined are} A, A.B, \emph{and} A.C.{]} Two
sub-entities shall not define classes with identical names {[}\emph{for
example, a directory shall not contain both the sub-directory} A
\emph{and the file} A.mo{]}.

In order to preserve the order of sub-entities it is advisable to create
a file package.order where each line contains the name of one class or
constant. If a package.order is present when reading a structured entity
the classes and constants are added in this order; if the contents does
not exactly match the classes and constants in the package, the
resulting order is tool specific and a warning may be given. Classes and
constants that are stored in package.mo are also present in
package.order but their relative order should be identical to the one in
package.mo (this ensures that the relative order between classes and
constants stored in different ways is preserved).

\subsubsection{Mapping a Package/Class Hierarchy into a Single File (Nonstructured Entity)}\doublelabel{mapping-a-package-class-hierarchy-into-a-single-file-nonstructured-entity}

When mapping a package or class-hierarchy to a file {[}\emph{e.g. the
  file} A.mo{]}, that file shall only define a single class {[}A.mo{]} with a
name matching the name of the nonstructured entity. In a file hierarchy
the files shall have the extension ``.mo''.

A ".mo" file defining more than one class cannot be part of the mapping
to file-structure and it is an error if it is loaded from the
MODELICAPATH

\subsubsection{The within Clause}\doublelabel{the-within-clause}

A within-clause has the following syntax:

\begin{lstlisting}[language=grammar]
  within [ packageprefixname ] ";"
\end{lstlisting}
  A non-top-level entity shall begin with a within-clause which for the
  class defined in the entity specifies the location in the Modelica class
    hierarchy. A top-level class may contain a within-clause with no name.
    For a sub-entity of an enclosing structured entity, the within-clause
shall designate the class of the enclosing entity; and this class must
exist and must not have been defined using a short class definition.

{[}\emph{Example: The subpackage} Rotational \emph{declared within}
Modelica.Mechanics \emph{has the fully qualified name}
Modelica.Mechanics.Rotational\emph{, which is formed by concatenating
the packageprefixname with the short name of the package. The
declaration of} Rotational \emph{could be given as below}:

\begin{lstlisting}[language=modelica]
  within Modelica.Mechanics;
  package Rotational // Modelica.Mechanics.Rotational
    ...
\end{lstlisting}
{]}

\subsection{External resources}\doublelabel{external-resources}

In order to reference external resources from documentation (such as
links and images in html-text) and/or to reference images in the Bitmap
annotation (see \ref{bitmap}). URIs should be used, for example
file:/// and the URI scheme modelica:// which can be used to retrieve
resources associated with a package. {[}\emph{Note scheme names are
case-insensitive, but the lower-case form should be used, that is
`}Modelica://\emph{' is allowed but `}modelica://\emph{' is the
recommended form.}{]}

The Modelica-scheme has the ability to reference a hierarchical
structure of resources associated with packages. The same structure is
used for all kind of resource references, independent of use (external
file, image in documentation, bitmap in icon layer, and link to external
file in the documentation), and regardless of the storage mechanism.

Any Modelica-scheme URI containing a slash after the package-name is
interpreted as a reference to a resource. The `authority' portion of the
URI is interpreted as a fully qualified package name and the path
portion of the URI is interpreted as the path (relative to the package)
of the resource. Each storage scheme can define its own interpretation
of the path (but care should be taken when converting from one storage
scheme or when restructuring packages that resource references resolve
to the same resource). Any storage scheme should be constrained such
that a resource with a given path should be unique for any package name
that precedes it. The first part of the path may not be the name of a
class in the package given by the authority.

When Modelica packages are stored hierarchically in a file-system (i.e.
package A in a directory A containing "package.mo") the resource
"modelica://A/Resources/C.jpg" should be stored in the file
"A/Resources/C.jpg", it is not recommend to use "modelica://A.B/C.jpg"
for referencing resources; it could be stored in the file "A/B/C.jpg" -
which is counter-intuitive if A.B is stored together with A. When
Modelica packages are stored in other formats a similar mapping should
be defined, such that a resource with a given path should be unique for
any package name that precedes it. The first part of the path may not be
the name of a class in the package given by the authority. As above for
"Modelica 3.2.1/package.mo" i.e. resources starting from "Modelica
3.2.1", and "modelica://Modelica.Mechanics/C.jpg" is "Modelica
3.2.1/Mechanics/C.jpg" - regardless of whether Modelica.Mechanics is
stored in "Modelica 3.2.1/package.mo", "Modelica
3.2.1/Mechanics/package.mo", or "Modelica 3.2.1/Mechanics.mo".

For a Modelica-package stored as a single file, "A.mo", the resource
"modelica://A/C.jpg" refers to a file "C.jpg" stored in the same
directory as "A.mo", but using resources in this variant is not
recommended since multiple packages will share resources.

In case the class-name contains quoted identifiers, the single-quote "`"
and any reserved characters (``:'', ``/'', ``?'', ``\#'', ``{[}``,
``{]}'', ``@'', ``!'', ``\$'', ``\&'', ``(``, ``)'', ``*'', ``+'',
``,'', ``;'', ``='') should be percent-encoded as normal in URIs.

{[}\emph{Example:}

\emph{Consider a top-level package} Modelica \emph{and a class}
Mechanics \emph{inside it, a reference such as}
modelica://Modelica.Mechanics/C.jpg \emph{is legal, while}
modelica://Modelica/Mechanics/C.jpg \emph{is illegal. The reference}
modelica://Modelica.Mechanics/C.jpg \emph{must also refer to a different
resource than} modelica://Modelica/C.jpg\emph{.}{]}

\subsection{The Modelica Library Path---MODELICAPATH}\doublelabel{the-modelica-library-path-modelicapath}

The top-level scope implicitly contains a number of classes stored
externally. If a top-level name is not found at global scope, a Modelica
translator shall look up additional classes in an ordered list of
library roots, called MODELICAPATH. {[}\emph{The implementation of}
MODELICAPATH \emph{is tool dependent. In order that a user can work in
parallel with different Modelica tools, it is advisable to not have this
list as environment variable, but as a setting in the respective tool.
Since MODELICAPATH is tool dependent, it is not specified in which way
the list of library roots is stored. Typically, on a Windows system
MODELICAPATH is a string with path names separated by ``;'' whereas on a
Linux system it is a string with path names separated by a ``:''.}{]}

In addition a tool may define an internal list of libraries, since it is
in general not advisable for a program installation to modify global
environment variables. The version information for a library (as defined
in \ref{annotations-for-version-handling}) may also be used during this search to search for a
specific version of the library (e.g. if Modelica library version 2.2 is
needed and the first directory in MODELICAPATH contain Modelica library
version 2.1, whereas the second directory contains Modelica version 2.2,
then Modelica library version 2.2 is loaded from the second directory.).

{[}\emph{The first part of the path} A.B.C (\emph{i.e.}, A) \emph{is
located by searching the ordered list of roots in} MODELICAPATH.
\emph{If no root contains} A \emph{the lookup fails}. \emph{If} A
\emph{has been found in one of the roots, the rest of the path is
located in} A; \emph{if that fails, the entire lookup fails without
searching} \emph{for} A \emph{in any of the remaining roots in}
MODELICAPATH.{]}

\subsubsection{Example of Searching MODELICAPATH}\doublelabel{example-of-searching-modelicapath}

If during lookup a top-level name is not found in the unnamed top-level
scope, the search continues in the package hierarchies stored in these
directories. {[}\emph{Figure 13-1 below shows an example MODELICAPATH =
"C:\textbackslash{}library;C:\textbackslash{}lib1;C:\textbackslash{}lib2",
with three directories containing the roots of the package hierarchies
Modelica, MyLib, and ComplexNumbers. The first two are represented as
the subdirectories C:\textbackslash{}library\textbackslash{}Modelica and
C:\textbackslash{}lib1\textbackslash{}MyLib, whereas the third is stored
as the file C:\textbackslash{}lib2\textbackslash{}ComplexNumbers.mo.}

Figure 13.1\textbf{. Roots of package hierarchies, e.g.,}
Modelica\textbf{,} MyLib\textbf{, and} ComplexNumbers \textbf{in}
MODELICAPATH =
"C:\textbackslash{}library;C:\textbackslash{}lib1;C:\textbackslash{}lib2"\textbf{.}

\emph{Assume that we want to access the package} MyLib.Pack2 \emph{in
Figure 13-1 above, e.g. through an} import \emph{clause} import
MyLib.Pack2;\emph{. During lookup we first try to find a package} MyLib
\emph{corresponding to the first part of the import name. It is not
found in the top-level scope since it has not previously been loaded
into the environment. }

\emph{Since the name was not found in the top-level scope the search
continues in the directories in the} MODELICAPATH \emph{in the specified
order. For the search to succeed, there must be a subdirectory} MyLib
\emph{or a file} MyLib.mo \emph{in one of the directories mentioned in
the} MODELICAPATH\emph{. If there is no such subdirectory or file, the
lookup fails. If} MyLib \emph{is found in one of the directories, the
rest of the name, in this case} Pack2\emph{, is located in} MyLib\emph{.
If that fails, the entire lookup fails without continuing the search in
possibly remaining directories.}

\emph{In this example the name matches the subdirectory named} MyLib
\emph{in the second directory} ``C:\textbackslash{}lib1''
\emph{mentioned in the} MODELICAPATH\emph{. This subdirectory must have
a file} package.mo \emph{containing a definition of the package}
MyLib\emph{, according to the Modelica rules on how to map a package
hierarchy to the file system. The subpackage} Pack2 \emph{is stored in
its own subdirectory or file in the subdirectory} MyLib\emph{. In this
case the search succeeds and the package} MyLib.Pack2 \emph{is loaded
into the environment.}{]}
