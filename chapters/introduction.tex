\chapter{Introduction}\doublelabel{introduction1}
\section{Overview of Modelica}\doublelabel{overview-of-modelica}
Modelica is a language for modeling of physical systems, designed to
support effective library development and model exchange. It is a modern
language built on acausal modeling with mathematical equations and
object-oriented constructs to facilitate reuse of modeling knowledge.

\section{Scope of the Specification}\doublelabel{scope-of-the-specification}

The semantics of the Modelica language is specified by means of a set of
rules for translating any class described in the Modelica language to a
flat Modelica structure.

A class (of specialized class model, class or block) intended to be
simulated on its own is called a simulation model.

The flat Modelica structure is also defined for other cases than
simulation models; including functions (can be used to provide
algorithmic contents), packages (used as a structuring mechanism), and
partial models (used as base-models). This allows correctness to be
verified for those classes, before using them to build the simulation
model.

There are specific semantic restrictions for a simulation model to
ensure that the model is complete; they allow its flat Modelica
structure to be further transformed into a set of differential,
algebraic and discrete equations (= flat hybrid DAE). Note that
satisfying the semantic restrictions does not guarantee that the model
can be initialized from the initial conditions and simulated.

Modelica was designed to facilitate symbolic transformations of models,
especially by mapping basically every Modelica language construct to
continuous or instantaneous equations in the flat Modelica structure.
Many Modelica models, especially in the associated Modelica Standard
Library, are higher index systems, and can only be reasonably simulated
if symbolic index reduction is performed, i.e., equations are
differentiated and appropriate variables are selected as states, so that
the resulting system of equations can be transformed to state space form
(at least locally numerically), i.e., a hybrid DAE of index zero. In
order to allow this structural analysis, a tool may reject simulating a
model if parameters cannot be evaluated during translation - due to
calls of external functions or initial equations/initial algorithms for
fixed=false parameters. Accepting such models is a quality of
implementation issue. The Modelica specification does not define how to
simulate a model. However, it defines a set of equations that the
simulation result should satisfy as well as possible.

The key issues of the translation (or flattening) are:

\begin{itemize}
\item
  Expansion of inherited base classes
\item
  Parameterization of base classes, local classes and components
\item
  Generation of connection equations from connect-equations
\end{itemize}

The flat hybrid DAE form consists of:

\begin{itemize}
\item
  Declarations of variables with the appropriate basic types, prefixes
  and attributes, such as "parameter Real v=5".
\item
  Equations from equation sections.
\item
  Function invocations where an invocation is treated as a set of
  equations which involves all input and all result variables (number of
  equations = number of basic result variables).
\item
  Algorithm sections where every section is treated as a set of
  equations which involves the variables occurring in the algorithm
  section (number of equations = number of different assigned
  variables).
\item
  When-clauses where every when-clause is treated as a set of
  conditionally evaluated equations, also called instantaneous
  equations, which are functions of the variables occurring in the
  clause (number of equations = number of different assigned variables).
\end{itemize}

Therefore, a flat hybrid DAE is seen as a set of equations where some of
the equations are only conditionally evaluated (e.g. instantaneous
equations are only evaluated when the corresponding when-condition
becomes true). Initial setup of the model is specified using
start-values and instantaneous equations that hold at the initial time
only.

A Modelica class may also contain annotations, i.e. formal comments,
which specify graphical representations of the class (icon and diagram),
documentation text for the class, and version information.

\section{Some Definitions}\doublelabel{some-definitions}

The semantic specification should be read together with the Modelica
grammar. Non-normative text, i.e., examples and comments, are enclosed
  in {[} {]}; comments are set in italics. Additional terms are explained
in the glossary in \autoref{glossary}. Some important terms are:

\begin{tabular}{|l|p{10cm}|}
\hline
\emph{Term} & \emph{Definition} \\
\hline
Component & An element defined by the production
component-clause in the Modelica grammar (basically a
variable or an instance of a class)\\
Element & Class definitions, extends-clauses and
component-clauses declared in a class (basically a class
reference or a component in a declaration).\\
Flattening & The translation of a model described in Modelica to the
corresponding model described as a hybrid DAE, involving expansion of
inherited base classes, parameterization of base classes, local classes
and components, and generation of connection equations from
connect-equations (basically, mapping the hierarchical structure of a
model into a set of differential, algebraic and discrete equations
together with the corresponding variable declarations and function
definitions from the model).\\
\hline
\end{tabular}

\section{Notation and Grammar}\doublelabel{notation-and-grammar}

The meta symbols (of the extended BNF-grammar) are defined in~\autoref{lexical-conventions}.

Boldface denotes keywords of the Modelica language. Keywords are
reserved words and may not be used as identifiers, with the exception of
initial which is a keyword in section headings, and der which is a
keyword for declaration functions, but it is also possible to call the
functions initial()and der(\ldots{}).

See \autoref{modelica-concrete-syntax} for a full lexical specification and grammar.