\chapter{Statements and Algorithm Sections}\label{statements-and-algorithm-sections}

Whereas equations are very well suited for physical modeling, there are
situations where computations are more conveniently expressed as
algorithms, i.e., sequences of statements. In this chapter we describe
the algorithmic constructs that are available in Modelica.

Statements are imperative constructs allowed in algorithm sections.

\section{Algorithm Sections}\label{algorithm-sections}

An algorithm section is comprised of the keyword \lstinline!algorithm! followed by a
sequence of statements. The formal syntax is as follows:
\begin{lstlisting}[language=grammar]
algorithm-section :
[ initial ] algorithm { statement ";" | annotation ";" }
\end{lstlisting}

Equation equality \lstinline!=! or any other kind of equation (see \cref{equations}) shall
not be used in an algorithm section.

\subsection{Initial Algorithm Sections}\label{initial-algorithm-sections}

See \cref{initialization-initial-equation-and-initial-algorithm} for a description of both initial algorithm sections and
initial equation sections.

\subsection{Execution of an algorithm in a model}\label{execution-of-an-algorithm-in-a-model}

An algorithm section is conceptually a code fragment that remains
together and the statements of an algorithm section are executed in the
order of appearance. Whenever an algorithm section is invoked, all
variables appearing on the left hand side of the assignment operator
\lstinline!:=! are initialized (at least conceptually):
\begin{itemize}
\item
  A continuous-time variable is initialized with its start value (i.e.\ the
  value of the \lstinline!start! attribute).
\item
  A discrete-time variable \lstinline!v! is initialized with \lstinline!pre(v)!.
\item
  If at least one element of an array appears on the left hand side of
  the assignment operator, then the complete array is initialized in
  this algorithm section.
\item
  A parameter assigned in an initial algorithm, \cref{initialization-initial-equation-and-initial-algorithm},
  is initialized with its start-value (i.e.\ the value of the \lstinline!start! attribute).
\end{itemize}

\begin{nonnormative}
Initialization is performed, in order that an algorithm section
cannot introduce a ``memory'' (except in the case of discrete-time variables assigned in the algorithm), which could invalidate the assumptions of a
numerical integration algorithm. Note, a Modelica tool may change the
evaluation of an algorithm section, provided the result is identical to
the case, as if the above conceptual processing is performed.

An algorithm section is treated as an atomic vector-equation,
which is sorted together with all other equations. For the sorting
process (BLT), every algorithm section with N different left-hand side
variables, is treated as an atomic N-dimensional vector-equation
containing all variables appearing in the algorithm section. This
guarantees that all N equations end up in an algebraic loop and the
statements of the algorithm section remain together.

Example:
\begin{lstlisting}[language=modelica]
model Test // wrong Modelica model (has 4 equations for 2 unknowns)
  Real x[2](start={-11, -22});
algorithm // conceptually: x = {1,-22}
  x[1] := 1;
algorithm // conceptually: x = {-11,2}
  x[2] := 2;
end Test;
\end{lstlisting}

The conceptual part indicate that if the variable is assigned unconditionally
in the algorithm before it is used the initialization can be omitted.
This is usually the case, except for algorithms with when-statements,
and especially for initial algorithms.
\end{nonnormative}

\subsection{Execution of the algorithm in a function}\label{execution-of-the-algorithm-in-a-function}

See \crefnameref{initialization-and-binding-equations-of-components-in-functions}.

\section{Statements}\label{statements}

Statements are imperative constructs allowed in algorithm sections. A
flattened statement is identical to the corresponding nonflattened
statement.

Names in statements are found as follows:
\begin{itemize}
\item
  If the name occurs inside an expression: it is first found among the
  lexically enclosing reduction functions (see \cref{reduction-functions-and-operators}) in order
  starting from the inner-most, and if not found it proceeds as if it
  were outside an expression:
\item
  Names in a statement are first found among the lexically enclosing
  for-statements in order starting from the inner-most, and if not
  found:
\item
  Names in a statement shall be found by looking up in the partially
  flattened enclosing class of the algorithm section.
\end{itemize}

The syntax of statements is as follows:
\begin{lstlisting}[language=grammar]
statement :
  ( component-reference ( ":=" expression | function-call-args )
    | "(" output-expression-list ")" ":=" component-reference function-call-args
    | break
    | return
    | if-statement
    | for-statement
    | while-statement
    | when-statement )
  comment
\end{lstlisting}

\subsection{Simple Assignment Statements}\label{simple-assignment-statements}

The syntax of simple assignment statement is as follows:
\begin{lstlisting}[language=grammar]
component-reference ":=" expression
\end{lstlisting}

The \lstinline!expression! is evaluated. The resulting value is stored into the
variable denoted by \lstinline!component-reference!.

\subsubsection{Assignments from Called Functions with Multiple Results}\label{assignments-from-called-functions-with-multiple-results}

There is a special form of assignment statement that is used only when
the right-hand side contains a call to a function with multiple results.
The left-hand side contains a parenthesized, comma-separated list of
variables receiving the results from the function call. A function with
\emph{n} results needs \emph{m\textless{}=n} receiving variables on the
left-hand side, and the variables are assigned from left to right.

\begin{lstlisting}[language=modelica]
(out1, out2, out3) := function_name(in1, in2, in3, in4);
\end{lstlisting}

It is possible to omit receiving variables from this list:
\begin{lstlisting}[language=modelica]
(out1,, out3) := function_name(in1, in2, in3, in4);
\end{lstlisting}

\begin{example}
The function \lstinline!f! called below has three results and two inputs:
\begin{lstlisting}[language=modelica]
(a, b, c) := f(1.0, 2.0);
(x[1], x[2], x[1]) := f(3, 4);
\end{lstlisting}
In the second example above \lstinline!x[1]! is assigned twice: first with the first output, and then with the third output.  For that case the following will give the same result:
\begin{lstlisting}[language=modelica]
(, x[2], x[1]) := f(3,4);
\end{lstlisting}
\end{example}

The syntax of an assignment statement with a call to a function with
multiple results is as follows:
\begin{lstlisting}[language=grammar]
"(" output-expression-list ")" ":=" component-reference function-call-args
\end{lstlisting}

\begin{nonnormative}
Also see \cref{simple-equality-equations} regarding calling functions with
multiple results within equations.
\end{nonnormative}

\subsubsection{Restrictions on assigned variables}\label{restrictions-on-assigned-variables}
Only components of the restricted classes \lstinline!type!, \lstinline!record!, \lstinline!operator record!, and \lstinline!connector! may appear as left-hand-side in algorithms.
This applies both to simple assignment statements, and the parenthesized, comma-separated list of variables for functions with multiple results.

\subsection{For-statement}\label{for-statement}

The syntax of a for-statement is as follows:
\begin{lstlisting}[language=grammar]
for for-indices loop
  { statement ";" }
end for
\end{lstlisting}
For-statements may optionally use several iterators (\lstinline!for-indices!), see
\cref{nested-for-loops-and-reduction-expressions-with-multiple-iterators} for more information:
\begin{lstlisting}[language=grammar]
for-indices:
   for-index {"," for-index}

for-index:
   IDENT [ in  expression ]
\end{lstlisting}
The following is an example of a prefix of a for-statement:
\begin{lstlisting}[language=modelica]
for IDENT in expression loop
\end{lstlisting}
The rules for for-statements are the same as for for-expressions in \cref{explicit-iteration-ranges-of-for-equations} -
except that the \lstinline!expression! of a for-statement is not restricted to a parameter-expression.

\begin{example}
\begin{lstlisting}[language=modelica]
for i in 1:10 loop // i takes the values 1,2,3,...,10
for r in 1.0 : 1.5 : 5.5 loop // r takes the values 1.0, 2.5, 4.0, 5.5
for i in {1,3,6,7} loop // i takes the values 1, 3, 6, 7
for i in TwoEnums loop // i takes the values TwoEnums.one, TwoEnums.two
        // for TwoEnums = enumeration(one,two)
\end{lstlisting}
The loop-variable may hide other variables as in the following
example. Using another name for the loop-variable is, however, strongly
recommended.
\begin{lstlisting}[language=modelica]
  constant Integer j=4;
  Real x[j];
equation
  for j in 1:j loop // The loop-variable j takes the values 1,2,3,4
    x[j]=j; // Uses the loop-variable j
  end for;
\end{lstlisting}
\end{example}

\subsubsection{Implicit Iteration Ranges}\label{implicit-iteration-ranges}

An iterator \lstinline!IDENT in range-expr! without the \lstinline!in range-expr! requires that
the \lstinline!IDENT! appears as the subscript of one or several subscripted
expressions, where the expressions are not part of an array in a component of an expandable connector.
The dimension size of the array expression in the indexed
position is used to deduce the \lstinline!range-expr! as
\lstinline!1:size(array-expression,indexpos)! if the indices are a subtype of
Integer, or as \lstinline!E.e1:E.en! if the indices are of an enumeration type
\lstinline!E=enumeration(e1, ..., en)!, or as \lstinline!false:true! if the indices are of
type Boolean. If it is used to subscript several expressions, their
ranges must be identical. The \lstinline!IDENT! may also, inside a
reduction-expression, array constructor expression, for-statement, or
for-equation, occur freely outside of subscript positions, but only as a
reference to the variable \lstinline!IDENT!, and not for deducing ranges.

The \lstinline!IDENT! may also be used as a subscript for an array in a component of an expandable connector
but it is only seen as a reference to the variable \lstinline!IDENT! and cannot be used for deducing ranges.

\begin{example}
\begin{lstlisting}[language=modelica]
  Real x[4];
  Real xsquared[:]={x[i]*x[i] for  i};
  // Same as: {x[i]*x[i] for i in 1:size(x,1)}
  Real xsquared2[size(x,1)];
  Real xsquared3[size(x,1)];
equation
  for i loop // Same as: for i in 1:size(x,1) loop ...
    xsquared2[i]=x[i]^2;
  end for;
algorithm
  for i loop // Same as: for i in 1:size(x,1) loop ...
    xsquared3[i] := x[i]^2;
  end for;
\end{lstlisting}

\begin{lstlisting}[language=modelica]
type FourEnums = enumeration(one, two, three, four);
Real xe[FourEnums] = x;
Real xsquared3[FourEnums] = {xe[i] * xe[i] for i};
Real xsquared4[FourEnums] = {xe[i] * xe[i] for i in FourEnums};
Real xsquared5[FourEnums] = {x[i] * x[i] for i};
\end{lstlisting}
\end{example}

The size of an array -- the iteration range -- is evaluated on entry to the for-loop and the array size shall not change during the execution of the for-loop.

\subsubsection{Types as Iteration Ranges }\label{types-as-iteration-ranges}

The iteration range can be specified as \lstinline!Boolean! or as an enumeration
type. This means iteration over the type from min to max, i.e.\ for
Boolean it is the same as \lstinline!false:true! and for an enumeration E it is the
same as E.min:E.max. This can be used for \lstinline!for! loops and reduction
expressions.

\begin{example}
\begin{lstlisting}[language=modelica]
  type FourEnums=enumeration(one,two,three,four);
  Real xe[FourEnums];
  Real xsquared1[FourEnums];
  Real xsquared2[FourEnums]={xe[i]*xe[i] for i in FourEnums};
equation
  for i in FourEnums loop
    xsquared1[i]=xe[i]^2;
  end for;
\end{lstlisting}
\end{example}

\subsubsection{Nested For-Loops and Reduction Expressions with Multiple Iterators}\label{nested-for-loops-and-reduction-expressions-with-multiple-iterators}

The notation with several iterators is a shorthand notation for nested
for-statements or for-equations (or reduction-expressions). For
for-statements or for-equations it can be expanded into the usual form
by replacing each `\lstinline!,!' by ``\lstinline!loop for!'' and adding extra ``\lstinline!end for!''. For
reduction-expressions it can be expanded into the usual form by
replacing each `\lstinline!,!' by ``\lstinline!) for!'' and prepending the reduction-expression
with ``\lstinline!functionName(!''.

\begin{example}
\begin{lstlisting}[language=modelica]
  Real x[4,3];
algorithm
  for j, i in 1:2 loop
    // The loop-variable j takes the values 1,2,3,4 (due to use)
    // The loop-variable i takes the values 1,2 (given range)
    x[j,i] := j+i;
  end for;
\end{lstlisting}
\end{example}

\subsection{While-Statement}\label{while-statement}

The while-statement has the following syntax:
\begin{lstlisting}[language=grammar]
while expression loop
  { statement ";" }
end while
\end{lstlisting}
The \lstinline!expression! of a while-statement shall be a scalar Boolean
expression. The while-statement corresponds to while-statements in
programming languages, and is formally defined as follows:
\begin{enumerate}
\item The \lstinline!expression! of the while-statement is evaluated.
\item If the \lstinline!expression! of the while-statement is false, the execution
continues after the while-statement.
\item If the \lstinline!expression! of the while-statement is true, the entire body of
the while-statement is executed (except if a break-statement, see
\cref{break-statement}, or a return-statement, see \cref{return-statements}, is executed),
and then execution proceeds at step 1.
\end{enumerate}

\subsection{Break-Statement}\label{break-statement}

The break-statement breaks the execution of the innermost while- or for-loop enclosing the break-statement and continues execution after the while- or for-loop.  It can only be used in a while- or for-loop in an algorithm section.  It has the following syntax:
\begin{lstlisting}[language=modelica]
break;
\end{lstlisting}

\begin{example}
(Note that this could alternatively use \lstinline!return!).
\begin{lstlisting}[language=modelica]
function findValue "Returns position of val or 0 if not found"
  input Integer x[:];
  input Integer val;
  output Integer index;
algorithm
  index := size(x, 1);
  while index >= 1 loop
    if x[index] == val then
      break;
    else
      index := index - 1;
    end if;
  end while;
end findValue;
\end{lstlisting}
\end{example}

\subsection{Return-Statements}\label{return-statements}

Can only be used inside functions, see \cref{function-return-statements}.

\subsection{If-Statement}\label{if-statement}

If-statements have the following syntax:
\begin{lstlisting}[language=grammar]
if expression then
  { statement ";" }
    { elseif expression then
  { statement ";" }
  }
  [ else
    { statement ";" }
  ]
end if;
\end{lstlisting}

The \lstinline!expression! of an if- or elseif-clause must be scalar Boolean
expression. One if-clause, and zero or more elseif-clauses, and an
optional else-clause together form a list of branches. One or zero of
the bodies of these if-, elseif- and else-clauses is selected, by
evaluating the conditions of the if- and elseif-clauses sequentially
until a condition that evaluates to true is found. If none of the
conditions evaluate to true the body of the else-clause is selected (if
an else-clause exists, otherwise no body is selected). In an algorithm
section, the selected body is then executed. The bodies that are not
selected have no effect on that model evaluation.

\subsection{When-Statements}\label{when-statements}

A when-statement has the following syntax:
\begin{lstlisting}[language=grammar]
when expression then
  { statement ";" }
  { elsewhen expression then
  { statement ";" } }
end when
\end{lstlisting}
The \lstinline!expression! of a when-statement shall be a discrete-time Boolean
scalar or vector expression. The algorithmic statements within a
when-statement are activated when the scalar or any one of the elements
of the vector-expression becomes true.

\begin{example}
Algorithms are activated when \lstinline!x! becomes \textgreater{} 2:
\begin{lstlisting}[language=modelica]
when x > 2 then
  y1 := sin(x);
  y3 := 2*x + y1+y2;
end when;
\end{lstlisting}
The statements inside the when-statement are activated on the positive edge of any of the expressions
\lstinline!x > 2!, \lstinline!sample(0, 2)!, or \lstinline!x < 5!:
\begin{lstlisting}[language=modelica]
when {x > 2, sample(0,2), x < 5} then
  y1 := sin(x);
  y3 := 2*x + y1+y2;
end when;
\end{lstlisting}
For when-statements in algorithm sections the order is significant
and it is advisable to have only one assignment within the
when-statement and instead use several algorithm sections having
when-statements with identical conditions, e.g.:
\begin{lstlisting}[language=modelica]
algorithm
  when x > 2 then
    y1 := sin(x);
  end when;
equation
  y2 = sin(y1);
algorithm
  when x > 2 then
    y3 := 2*x +y1+y2;
  end when;
\end{lstlisting}
Merging the when-statements can lead to less efficient code and
different models with different behavior depending on the order of the
assignment to \lstinline!y1! and \lstinline!y3! in the algorithm.
\end{example}

\subsubsection{Restrictions on When-Statements}\label{restrictions-on-when-statements}

\begin{itemize}
\item
  A when-statement shall not be used within a function.
\item
  When-statements shall not occur inside initial algorithms.
\item
  When-statements cannot be nested.
\item
  When-statements shall not occur inside while, if, and for-clauses in
  algorithms.
\end{itemize}

\begin{example}
The following nested when-statement is invalid:
\begin{lstlisting}[language=modelica]
when x > 2 then
  when y1 > 3 then
    y2 := sin(x);
  end when;
end when;
\end{lstlisting}
\end{example}

\subsubsection{Defining When-Statements by If-Statements}\label{defining-when-statements-by-if-statements}

A when-statement:
\begin{lstlisting}[language=modelica]
algorithm
  when {x>1, ..., y>p} then
    ...
    elsewhen x > y.start then
    ...
  end when;
\end{lstlisting}
is similar to the following special if-statement, where \lstinline!Boolean b1[N];! and \lstinline!Boolean b2;! are necessary because \lstinline!edge! can
only be applied to variables
\begin{lstlisting}[language=modelica]
  Boolean b1[N](start={x.start>1, ...,
  y.start>p});
  Boolean b2(start=x.start>y.start);
algorithm
  b1:={x>1, ..., y>p};
  b2:=x>y.start;
  if edge(b1[1]) or edge(b1[2]) or ...
    edge(b1[N]) then
    ...
    elseif edge(b2) then
    ...
  end if;
\end{lstlisting}
with \lstinline!edge(A)= A and not pre(A)! and the additional guarantee, that the
statements within this special if-statement are only evaluated at event
instants. The difference compared to the when-statements is that e.g.\ \lstinline!pre! may only be used on continuous-time real variables inside the body
of a when-clause and not inside these if-statements.

\subsection{Special Statements}\label{special-statements}

These special statements have the same form and semantics as the
corresponding equations, apart from the general difference in semantics
between equations and statements.

\subsubsection{Assert Statement}\label{assert-statement}

See \cref{assert}.  A failed \lstinline!assert! stops the execution of the current algorithm.

\subsubsection{Terminate Statement}\label{terminate-statement}

See \cref{terminate}.  The \lstinline!terminate! statement shall not be in functions.  In an algorithm outside a function it does not stop the execution of the current algorithm.
