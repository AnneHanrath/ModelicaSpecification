\chapter{Modelica Concrete Syntax}\doublelabel{modelica-concrete-syntax}
\section{Lexical conventions}\doublelabel{lexical-conventions}

The following syntactic meta symbols are used (extended BNF):
\begin{lstlisting}[language=grammar]
[ ] optional
{ } repeat zero or more times
| or
"text" The text is treated as a single token (no whitespace between any characters)
\end{lstlisting}

The following lexical units are defined (the ones in boldface are the
ones used in the grammar, the rest are just internal to the definition
of other lexical units):

\begin{lstlisting}[language=grammar]
IDENT = NONDIGIT { DIGIT | NONDIGIT } | Q-IDENT
Q-IDENT = "'" ( Q-CHAR | S-ESCAPE ) { Q-CHAR | S-ESCAPE | """ } "'"
NONDIGIT = "_" | letters "a" to "z" | letters "A" to "Z"
STRING = """ { S-CHAR | S-ESCAPE } """
S-CHAR = see below
Q-CHAR = NONDIGIT | DIGIT | "!" | "#" | "$" | "%" | "&" | "(" | ")" | "*" | "+" | "," |
         "-" | "." | "/" | ":" | ";" | "<" | ">" | "=" | "?" | "@" | "[" | "]" | "^" |
        "{" | "}" | "|" | "~" | " "
S-ESCAPE = "\'" | "\"" | "\?" | "\\" |
         "\a" | "\b" | "\f" | "\n" | "\r" | "\t" | "\v"
DIGIT = "0" | "1" | "2" | "3" | "4" | "5" | "6" | "7" | "8" | "9"
UNSIGNED-INTEGER = DIGIT { DIGIT }
UNSIGNED-REAL = UNSIGNED-INTEGER  "." [ UNSIGNED-INTEGER ]
     |  UNSIGNED_INTEGER [ "." [ UNSIGNED_INTEGER ] ] ( "e" | "E" ) [ "+" | "-" ] UNSIGNED-INTEGER
     |   "."  UNSIGNED-INTEGER [ ( "e" | "E" ) [ "+" | "-" ] UNSIGNED-INTEGER ]
\end{lstlisting}
\textrm{S-CHAR} is any member of the Unicode character set
(\url{http://www.unicode.org}; see \autoref{mapping-package-class-structures-to-a-hierarchical-file-system} for storing as UTF-8 on files) except double-quote """, and backslash "\textbackslash{}"

For identifiers the redundant escapes (`\lstinline!\?!' and `\lstinline!\"!') are the same as the corresponding non-escaped
variants (`\lstinline!?!' and '\lstinline!"!'). {[}\emph{The single quotes are part of an
identifier. E.g. \lstinline!'x'! and \lstinline!x! are different IDENTs}{]}.

Note:
\begin{itemize}
\item
  Whitespace and comments can be used between separate lexical units
  and/or symbols, and also separates them. Each lexical unit will consume the maximum number of characters from the input stream.
  Whitespace and comments
  cannot be used inside other lexical units, except for STRING and
  Q-IDENT where they are treated as part of the STRING or Q-IDENT
  lexical unit.
\item
  String constant concatenation \lstinline!"a" "b"! becoming \lstinline!"ab"! (as in C) is
  replaced by the \lstinline!"+"! operator in Modelica.
\item
  Modelica uses the same comment syntax as C++ and Java (i.e., \lstinline!//!
  signals the start of a line comment and \lstinline!/*! .... \lstinline!*/! is a multi-line
  comment); comments may contain any Unicode character. Modelica also
  has structured comments in the form of annotations and string
  comments.
\item
  Each description-string or string in annotations (= STRING with production annotation-clause in the
  grammar) may contain any member of the Unicode character set. All
  other strings have to contain only the sub-set of Unicode characters
  identical with the 7-bit US-ASCII character set. {[}\emph{As a
  consequence, operators like `\textgreater{}' or `\textless{}', and
  external functions only operate on ASCII strings and not on
  Unicode-strings.} \emph{Within a description-string the tags
  \lstinline!<HTML>! and \lstinline!</HTML>! or
  \lstinline!<html>! and \lstinline!</html>!
  define optionally begin and end of content that is HTML encoded.}{]}
\item
  Boldface denotes keywords of the Modelica language. Keywords are
  reserved words and may not be used as identifiers.
\item
  Productions use hyphen as separator both in the grammar and in the
  text. Previously the grammar used underscore.
\end{itemize}

\section{Grammar}\doublelabel{grammar}
\subsection{Stored Definition -- Within}\doublelabel{stored-definition-within}
\begin{lstlisting}[language=grammar]
stored-definition:
   [ within [ name ] ";" ]
   { [ final ] class-definition ";" }
\end{lstlisting}

\subsection{Class Definition}\doublelabel{class-definition}
\begin{lstlisting}[language=grammar]
class-definition :
   [ encapsulated ] class-prefixes
   class-specifier

class-prefixes :
   [ partial ]
   ( class | model | [ operator ] record | block | [ expandable ] connector | type |
   package | [ pure | impure ] [ operator ] function | operator )

class-specifier :
   long-class-specifier | short-class-specifier | der-class-specifier

long-class-specifier :
   IDENT description-string composition end IDENT
   | extends IDENT [ class-modification ] description-string composition
   end IDENT

short-class-specifier :
   IDENT "=" base-prefix type-specifier [ array-subscripts ]
   [ class-modification ] comment
   | IDENT "=" enumeration "(" ( [ enum-list ] | ":" ) ")" comment

der-class-specifier :
   IDENT "=" der "(" type-specifier "," IDENT { "," IDENT } ")" comment

base-prefix :
   [ input | output ]

enum-list : enumeration-literal { "," enumeration-literal}

enumeration-literal : IDENT comment

composition :
   element-list
   { public element-list |
     protected element-list |
     equation-section |
     algorithm-section
   }
   [ external [ language-specification ]
   [ external-function-call ] [ annotation-clause ] ";" ]
   [ annotation-clause ";" ]

language-specification :
   STRING

external-function-call :
   [ component-reference "=" ]
   IDENT "(" [ expression-list ] ")"

element-list :
   { element ";" }

element :
   import-clause |
   extends-clause |
   [ redeclare ]
   [ final ]
   [ inner ] [ outer ]
   ( class-definition | component-clause |
   replaceable ( class-definition | component-clause )
   [ constraining-clause comment ] )

import-clause :
   import ( IDENT "=" name | name [ ".*" | "." ( "*" | "{" import-list "}" ) ] ) comment

import-list :
   IDENT { "," IDENT }
\end{lstlisting}

\subsection{Extends}\doublelabel{extends}
\begin{lstlisting}[language=grammar]
extends-clause :
   extends type-specifier [ class-modification ] [ annotation-clause ]

constraining-clause :
   constrainedby type-specifier [ class-modification ]
\end{lstlisting}

\subsection{Component Clause}\doublelabel{component-clause}
\begin{lstlisting}[language=grammar]
component-clause:
   type-prefix type-specifier [ array-subscripts ] component-list

type-prefix :
   [ flow | stream ]
   [ discrete | parameter | constant ] [ input | output ]
component-list :
   component-declaration { "," component-declaration }

component-declaration :
   declaration [ condition-attribute ] comment

condition-attribute:
   if expression

declaration :
   IDENT [ array-subscripts ] [ modification ]
\end{lstlisting}

\subsection{Modification}\doublelabel{modification}
\begin{lstlisting}[language=grammar]
modification :
   class-modification [ "=" expression ]
   | "=" expression
   | ":=" expression

class-modification :
   "(" [ argument-list ] ")"

argument-list :
   argument { "," argument }

argument :
   element-modification-or-replaceable
   | element-redeclaration

element-modification-or-replaceable:
   [ each ] [ final ] ( element-modification | element-replaceable )

element-modification :
   name [ modification ] description-string

element-redeclaration :
   redeclare [ each ] [ final ]
   ( short-class-definition | component-clause1 | element-replaceable )

element-replaceable:
   replaceable ( short-class-definition | component-clause1 )
   [ constraining-clause ]

component-clause1 :
   type-prefix type-specifier component-declaration1

component-declaration1 :
   declaration comment

short-class-definition :
   class-prefixes short-class-specifier
\end{lstlisting}

\subsection{Equations}\doublelabel{equations1}
\begin{lstlisting}[language=grammar]
equation-section :
   [ initial ] equation { equation ";" }

algorithm-section :
   [ initial ] algorithm { statement ";" }

equation :
   ( simple-expression "=" expression
     | if-equation
     | for-equation
     | connect-clause
     | when-equation
     | component-reference function-call-args )
   comment

statement :
   ( component-reference ( ":=" expression | function-call-args )
     | "(" output-expression-list ")" ":=" component-reference function-call-args
     | break
     | return
     | if-statement
     | for-statement
     | while-statement
     | when-statement )
   comment

if-equation :
   if expression then
     { equation ";" }
   { elseif expression then
     { equation ";" }
   }
   [ else
     { equation ";" }
   ]
   end if

if-statement :
   if expression then
     { statement ";" }
   { elseif expression then
     { statement ";" }
   }
   [ else
     { statement ";" }
   ]
   end if

for-equation :
   for for-indices loop
     { equation ";" }
   end for

for-statement :
   for for-indices loop
     { statement ";" }
   end for

for-indices :
   for-index {"," for-index}

for-index:
   IDENT [ in expression ]

while-statement :
   while expression loop
   { statement ";" }
   end while

when-equation :
   when expression then
     { equation ";" }
   { elsewhen expression then
     { equation ";" } }
   end when

when-statement :
   when expression then
     { statement ";" }
   { elsewhen expression then
     { statement ";" } }
   end when

connect-clause :
   connect "(" component-reference "," component-reference ")"
\end{lstlisting}
\subsection{Expressions}\doublelabel{expressions1}
\begin{lstlisting}[language=grammar]
expression :
   simple-expression
   | if expression then expression { elseif expression then expression }
   else expression

simple-expression :
   logical-expression [ ":" logical-expression [ ":" logical-expression ] ]

logical-expression :
   logical-term { or logical-term }

logical-term :
   logical-factor { and logical-factor }

logical-factor :
   [ not ] relation

relation :
   arithmetic-expression [ relational-operator arithmetic-expression ]

relational-operator :
   "<" | "<=" | ">" | ">=" | "==" | "<>"

arithmetic-expression :
   [ add-operator ] term { add-operator term }

add-operator :
   "+" | "-" | ".+" | ".-"

term :
   factor { mul-operator factor }

mul-operator :
   "*" | "/" | ".*" | "./"

factor :
   primary [ ("^" | ".^") primary ]

primary :
   UNSIGNED-NUMBER
   | STRING
   | false
   | true
   | ( component-reference | der | initial | pure ) function-call-args
   | component-reference
   | "(" output-expression-list ")"
   | "[" expression-list { ";" expression-list } "]"
   | "{" array-arguments "}"
   | end

UNSIGNED-NUMBER : UNSIGNED-INTEGER | UNSIGNED-REAL

type-specifier : ["."] name

name : IDENT { "." IDENT }

component-reference :
   [ "." ] IDENT [ array-subscripts ] { "." IDENT [ array-subscripts ] }

function-call-args :
   "(" [ function-arguments ] ")"

function-arguments :
expression [ "," function-arguments-non-first | for for-indices ]
   | function type-specifier "(" [ named-arguments ] ")" [ "," function-arguments-non-first ]
   | named-arguments

function-arguments-non-first :
   function-argument [ "," function-arguments-non-first ]
   | named-arguments

array-arguments :
   expression [ "," array-arguments-non-first | for for-indices ]

array-arguments-non-first :
   expression [ "," array-arguments-non-first ]

named-arguments: named-argument [ "," named-arguments ]

named-argument: IDENT "=" function-argument

function-argument :
   function type-specifier "(" [ named-arguments ] ")" | expression

output-expression-list:
   [ expression ] { "," [ expression ] }

expression-list :
   expression { "," expression }

array-subscripts :
   "[" subscript { "," subscript } "]"

subscript :
   ":" | expression

description :
   description-string [ annotation-clause ]

description-string :
[ STRING { "+" STRING } ]

annotation-clause :
   annotation class-modification
\end{lstlisting}
