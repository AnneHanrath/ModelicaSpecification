\chapter{Unit Expressions}\doublelabel{unit-expressions}

Unless otherwise stated, the syntax and semantics of unit expressions in
Modelica are conform with the international standards ISO 31/0-1992
"General principles concerning quantities, units and symbols" and ISO
1000-1992 "SI units and recommendations for the use of their multiples
and of certain other units". Unfortunately, neither these two standards
nor other existing or emerging ISO standards define a formal syntax for
unit expressions. There are recommendations and Modelica exploits them.

Examples for the syntax of unit expressions used in Modelica: "\lstinline!N.m!",
"\lstinline!kg.m/s2!", "\lstinline!kg.m.s-2!" "\lstinline!1/rad!", 
"\lstinline!mm/s!".

\section{The Syntax of Unit Expressions}\doublelabel{the-syntax-of-unit-expressions}
\begin{lstlisting}[language=grammar]
unit_expression:
   unit_numerator [ "/" unit_denominator ]
   
unit_numerator:
   "1" | unit_factors | "(" unit_expression ")"
   
unit_denominator:
   unit_factor | "(" unit_expression ")"
\end{lstlisting}

The unit of measure of a dimension free quantity is denoted by "1". The
ISO standard does not define any precedence between multiplications and
divisions. The ISO recommendation is to have at most one division, where
the expression to the right of "\lstinline!/!" either contains no multiplications or
is enclosed within parentheses. It is also possible to use negative
exponents, for example, "\lstinline!J/(kg.K)!" may be written as "\lstinline!J.kg-1.K-1!".

\begin{lstlisting}[language=grammar]
unit_factors:
   unit_factor [ unit_mulop unit_factors ]

unit_mulop:
   "."
\end{lstlisting}

The ISO standard allows that a multiplication operator symbol is left
out. However, Modelica enforces the ISO recommendation that each
multiplication operator is explicitly written out in formal
specifications. For example, Modelica does not support "\lstinline!Nm!" for
newtonmeter, but requires it to written as "\lstinline!N.m!".

The preferred ISO symbol for the multiplication operator is a "dot" a
bit above the base line: "·". Modelica supports the ISO alternative ".",
which is an ordinary "dot" on the base line.

\begin{lstlisting}[language=grammar]
unit_factor:
  unit_operand [ unit_exponent ]

unit_exponent:
   [ "+" | "-" ] integer
\end{lstlisting}

The ISO standard does not define any operator symbol for exponentiation.
A \lstinline!unit_factor! consists of a \lstinline!unit_operand! possibly suffixed by a
possibly signed integer number, which is interpreted as an exponent.
There must be no spacing between the \lstinline!unit_operand! and a possible
\lstinline!unit_exponent!.

\begin{lstlisting}[language=grammar]
unit_operand:
   unit_symbol | unit_prefix unit_symbol

unit_prefix:
   Y | Z | E | P | T | G | M | k | h | da | d | c | m | u | n | p | f | a | z | y
\end{lstlisting}

A \lstinline!unit_symbol! is a string of letters. A basic support of units in
Modelica should know the basic and derived units of the SI system. It is
possible to support user defined unit symbols. In the base version Greek
letters is not supported, but full names must then be written, for
example "\lstinline!Ohm!".

A \lstinline!unit_operand! should first be interpreted as a \lstinline!unit_symbol! and only
if not successful the second alternative assuming a prefixed operand
should be exploited. There must be no spacing between the \lstinline!unit_symbol!
and a possible \lstinline!unit_prefix!. The values of the prefixes are according to
the ISO standard. The letter "u" is used as a symbol for the prefix
micro.

\begin{nonnormative}
A tool may present \lstinline!Ohm! as $\Omega$ and the prefix \lstinline!u! as $\mu$, and similarly \lstinline!m2! as $m^2$. 
\end{nonnormative}

\section{Examples}\doublelabel{examples2}

The unit expression "\lstinline!m!" means meter and not milli
(10\textsuperscript{-3}), since prefixes cannot be used in isolation.
For millimeter use "\lstinline!mm!" and for squaremeter, m\textsuperscript{2}, write
"\lstinline!m2!".

The expression "\lstinline!mm2!" means \lstinline!mm2! = (10\textsuperscript{-3}m)2 =
10\textsuperscript{-6}m\textsuperscript{2}. Note that exponentiation
includes the prefix.

The unit expression "\lstinline!T!" means Tesla, but note that the letter "\lstinline!T!" is
also the symbol for the prefix \lstinline!tera! which has a multiplier value of
10\textsuperscript{12}.
