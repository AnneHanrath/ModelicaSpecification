\chapter{Glossary}\doublelabel{glossary}
\textbf{algorithm section}: part of a class definition consisting of the
keyword \lstinline!algorithm! followed by a sequence of statements. Like an
equation, an algorithm section relates variables, i.e. constrains the
values that these variables can take simultaneously. In contrast to an
equation section, an algorithm section distinguishes inputs from
outputs: An algorithm section specifies how to compute output variables
as a function of given input variables. A Modelica processor may
actually invert an algorithm section, i.e. compute inputs from given
outputs, e.g. by search (generate and test), or by deriving an inverse
algorithm symbolically. (See \autoref{statements-and-algorithm-sections}.)

\textbf{array} or array variable: a component whose components are array
elements. For an array, the ordering of its components matters: The kth
element in the sequence of components of an array x is the array element
  with index \lstinline!k!, denoted \lstinline!x[k]!. All elements of an array have the same
  type. An array element may again be an array, i.e. arrays can be nested.
An array element is hence referenced using n indices in general, where n
is the number of dimensions of the array. Special cases are matrix (n=2)
and vector (n=1). Integer indices start with 1, not zero. (See \autoref{arrays}.)

\textbf{array constructor}: an array can be built using the
array-function -- with the shorthand \{a, b, \ldots{}\}, and can also
include an iterator to build an array of expressions. (See \autoref{vector-matrix-and-array-constructors}.)

\textbf{array element}: a component contained in an array. An array
element has no identifier. Instead they are referenced by array access
expressions called indices that use enumeration values or positive
integer index values. (See \autoref{arrays}.)

\textbf{assignment}: a statement of the form \lstinline!x := expr!. The expression
\lstinline!expr! must not have higher variability than~x. (See \autoref{simple-assignment-statements}.)

\textbf{attribute}: a component contained in a scalar component, such as
\lstinline!min!, \lstinline!max!, and \lstinline!unit!. All attributes are predefined and attribute values
can only be defined using a modification, such as in \lstinline!Real x(unit="kg")!.
Attributes cannot be accessed using dot notation, and are not
constrained by equations and algorithm sections. E.g. in \lstinline!Real x(unit="kg") = y;! only the values of \lstinline!x! and 
\lstinline!y! are declared to be equal,
but not their unit attributes, nor any other attribute of \lstinline!x! and \lstinline!y!. (See
\autoref{predefined-types-and-classes}.)

\textbf{base class}: class A is called a base class of B, if class B
extends class A. This relation is specified by an extends clause in B or
in one of B's base classes. A class inherits all elements from its base
classes, and may modify all non-final elements inherited from base
classes. (See \autoref{inheritance-extends-clause}.)

\textbf{binding equation}: Either a declaration equation or an element
modification for the value of the variable. A component with a binding
equation has its value bound to some expression. (See \autoref{equation-categories}.)

\textbf{class}: a description that generates an object called instance.
The description consists of a class definition, a modification
environment that modifies the class definition, an optional list of
dimension expressions if the class is an array class, and a lexically
enclosing class for all classes. (See \autoref{class-declarations}.)

\textbf{class type} or \textbf{inheritance interface}: property of a
class, consisting of a number of attributes and a set of public or
protected elements consisting of element name, element type, and element
attributes. (See \autoref{inheritance-interface-or-class-type}.)

\textbf{component} or \textbf{variable}: an instance (object) generated
by a component declaration. Special kinds of components are scalar,
array, and attribute. (See \autoref{component-declarations}.)

\textbf{component declaration}: an element of a class definition that
generates a component. A component declaration specifies (1) a component
name, i.e., an identifier, (2) the class to be flattened in order to
generate the component, and (3) an optional Boolean parameter
expression. Generation of the component is suppressed if this parameter
expression evaluates to false. A component declaration may be overridden
by an element-redeclaration. (See \autoref{component-declarations}.)

\textbf{component reference}: An expression containing a sequence of
identifiers and indices. A component reference is equivalent to the
referenced object, which must be a component. A component reference is
resolved (evaluated) in the scope of a class (or expression for the case
of a local iterator variable). A scope defines a set of visible
  components and classes. Example reference: \lstinline!Ele.Resistor.u[21].r! (See
\autoref{component-declarations} and \autoref{slice-operation}.)

\textbf{declaration assignment}: assignment of the form \lstinline!x := expression!
defined by a component declaration. This is similar to a declaration
equation. In contrast to a declaration equation, a declaration
assignment is allowed only when declaring a component contained in a
function. (See \autoref{initialization-and-declaration-assignments-of-components-in-functions}.)

\textbf{declaration equation}: Equation of the form \lstinline!x = expression!
defined by a component declaration. The expression must not have higher
variability than the declared component x. Unlike other equations, a
declaration equation can be overridden (replaced or removed) by an
element modification. (See \autoref{declaration-equations}.)

\textbf{derived class} or \textbf{subclass} or \textbf{extended class}:
class B is called derived from A, if B extends A. (See \autoref{inheritance-modification-and-redeclaration}.)

\textbf{element}: part of a class definition, one of: class definition,
component declaration, or extends clause. Component declarations and
class definitions are called named elements. An element is either
inherited from a base class or local.

\textbf{element modification}: part of a modification, overrides the
declaration equation in the class used by the instance generated by the
modified element. Example: \lstinline!vcc(unit="V")=1000!. (See \autoref{modifications}.)

\textbf{element-redeclaration}: part of a modification, replaces one of
the named elements possibly used to build the instance geneated by the
element that contains the redeclaration. Example: \lstinline!redeclare type Voltage = Real(unit="V")! replaces \lstinline!type Voltage!. (See \autoref{redeclaration}.)

\textbf{encapsulated}: a class that does not depend on where it is
placed in the package-hierarchy, since its lookup is stopped at the
encapsulated boundary. (See \autoref{simple-name-lookup}).

\textbf{equation}: part of a class definition. A scalar equation relates
scalar variables, i.e. constrains the values that these variables can
take simultaneously. When n-1 variables of an equation containing n
variables are known, the value of the nth variable can be inferred
(solved for). In contrast to a statement in an algorithm section, an
equation does not define for which of its variable it is to be solved.
Special cases are: initial equations, instantaneous equations,
declaration equations. (See \autoref{equations}.)

\textbf{event}: something that occurs instantaneously at a specific time
or when a specific condition occurs. Events are for example defined by
the condition occurring in a when clause, if clause, or if expression.
(See \autoref{events-and-synchronization}.)

\textbf{expression}: a term built from operators, function references,
components, or component references (referring to components) and
literals. Each expression has a type and a variability. (See \autoref{operators-and-expressions}.)

\textbf{extends clause}: an unnamed element of a class definition that
uses a name and an optional modification to specify a base class of the
class defined using the class definition. (See \autoref{inheritance-modification-and-redeclaration}.)

\textbf{flattening}: the computation that creates a flattened class of a
given class, where all inheritance, modification, etc. has been
performed and all names resolved, consisting of a flat set of equations,
algorithm sections, component declarations, and functions. (See \autoref{flattening-process}.)

\textbf{function}: a class of the specialized class function. (See \autoref{functions}.)

\textbf{function subtype} or \textbf{function compatible interface}: A
is a function subtype of B iff A is a subtype of B and the additional
arguments of function A that are not in function B are defined in such a
way (e.g. additional arguments need to have default values), that A can
be called at places where B is called. (See \autoref{function-compatibility-or-function-subtyping-for-functions}.)

\textbf{identifier} or ident: an atomic (not composed) name. Example:
\lstinline!Resistor! (See \autoref{identifiers-names-and-keywords}.)

\textbf{index} or \textbf{subscript}: An expression, typically of
Integer type or the colon symbol (:), used to reference a component (or
a range of components) of an array. (See \autoref{array-indexing}.)

\textbf{inheritance interface} or \textbf{class type}: property of a
class, consisting of a number of attributes and a set of public or
protected elements consisting of element name, element type, and element
attributes. (See \autoref{inheritance-interface-or-class-type}.)

\textbf{instance}: the object generated by a class. An instance contains
zero or more components (i.e. instances), equations, algorithms, and
local classes. An instance has a type. Basically, two instances have
same type, if their important attributes are the same and their public
components and classes have pair wise equal identifiers and types. More
specific type equivalence definitions are given e.g. for functions.

\textbf{instantaneous}: An equation or statement is instantaneous if it
holds only at events, i.e., at single points in time. The equations and
statements of a when-clause are instantaneous. (See \autoref{when-equations} and
\autoref{when-statements}.)

\textbf{interface}: see type. (See \autoref{interface-or-type}.)

\textbf{literal}: a real, integer, boolean, enumeration, or string
literal. Used to build expressions. (See \autoref{literal-constants}.)

\textbf{matrix}: an array where the number of dimensions is 2. (See
\autoref{arrays}.)

\textbf{modification}: part of an element. Modifies the instance
generated by that element. A modification contains element modifications
and element redeclarations. (See \autoref{modifications}.)

\textbf{modification environment}: the modification environment of a
class defines how to modify the corresponding class definition when
flattening the class. (See \autoref{modification-environment}.)

\textbf{name}: Sequence of one or more identifiers. Used to reference a
class. A class name is resolved in the scope of a class, which defines a
set of visible classes. Example name: "\lstinline!Ele.Resistor!". (See \autoref{names}.)

\textbf{operator record}: A record with user-defined operations;
defining e.g. multiplication and addition see \autoref{overloaded-operators}.

\textbf{partial}: a class that is incomplete and cannot be instantiated
in a simulation model; useful e.g. as a base-class. (See \autoref{component-declaration-static-semantics}.)

\textbf{partial flattening}: first find the names of declared local
classes and components. Modifiers, if present, are merged to the local
elements and redeclarations are performed. Then base-classes are looked
up, flattened and inserted into the class. See also flattening, which
additionally flattens local elements and performs modifications. (See
\autoref{flattening-process}.)

\textbf{plug-compatibility}: see restricted subtyping and \autoref{plug-compatibility-or-restricted-subtyping}.

\textbf{predefined type}: one of the types \lstinline!Real!, \lstinline!Boolean!, \lstinline!Integer!,
\lstinline!String! and types defined as \lstinline!enumeration! types. The component
declarations of the predefined types define attributes such as \lstinline!min!, \lstinline!max!,
and \lstinline!unit!. (See \autoref{predefined-types-and-classes}.)

\textbf{prefix}: property of an element of a class definition which can
be present or not be present, e.g. \lstinline!final!, \lstinline!public!, \lstinline!flow!. (See \autoref{prefix-rules}.)

\textbf{primitive type}: one of the built-in types \lstinline!RealType!,
\lstinline!BooleanType!, \lstinline!IntegerType!, \lstinline!StringType!, \lstinline!EnumType!. The primitive types are
used to define attributes and value of predefined types and enumeration
types. (See \autoref{predefined-types-and-classes}.)

\textbf{redeclare}: the modifier that changes a replaceable element.
(See \autoref{redeclaration})

\textbf{replaceable}: an element that can be replaced by a different
element having a compatible interface. (See \autoref{redeclaration})

\textbf{restricted subtyping} or \textbf{plug-compatibility}: a type A
is a restricted subtype of type B iff A is a subtype of B, and all
public components present in A but not in B must be default-connectable.
This is used to avoid introducing, via a redeclaration, an un-connected
connector in the object/class of type A at a level where a connection is
not possible. (See \autoref{plug-compatibility-or-restricted-subtyping}.)

\textbf{scalar} or scalar variable: a variable that is not an array.

\textbf{simple type:} Real, Boolean, Integer, String and enumeration
types

\textbf{specialized class}: one of: model, connector, package, record,
block, function, type. The class restriction of a class represents an
assertion regarding the content of the class and restricts its use in
other classes. For example, a class having the package class restriction
must only contain classes and constants. (See \autoref{specialized-classes}.)

\textbf{subtype} or \textbf{interface compatible}: relation between
types. A is a subtype of (interface compatible with) B iff a number of
properties of A and B are the same and all important elements of B have
corresponding elements in A with the same names and their types being
subtypes of the corresponding element types in B. See also restricted
subtyping and function restricted subtyping. (See \autoref{interface-compatibility-or-subtyping}.)

\textbf{supertype}: relation between types. The inverse of subtype. A is
a subtype of B means that B is a supertype of A. (See \autoref{interface-compatibility-or-subtyping}.)

\textbf{transitively nonreplaceable}: a class reference is considered
transitively non-replaceable if there are no replaceable elements in the
referenced class, or any of its base classes or constraining types
transitively at any level. (See \autoref{transitively-non-replaceable}.)

\textbf{type} or interface: property of an instance, expression, consisting of a number of attributes and a set of public
elements consisting of element name, element type, and element
attributes. Note: The concept of class type is a property of a class
definition. (See \autoref{interface-or-type}.)

\textbf{variability}: property of an expression: one of
\begin{itemize}
\item \textbf{continuous}: an expression that may change its value at any
point in time.
\item \textbf{discrete}: may change its value only at events during
simulation.
\item \textbf{parameter}: constant during the entire simulation, recommended
to change for a component.
\item \textbf{constant}: constant during the entire simulation (can be used
in a package) .
\end{itemize}

Assignments x := expr and binding equations x = expr must satisfy a
variability constraint: The expression must not have a higher
variability than component x. (See \autoref{variability-of-expressions}.)

\textbf{variable}: synonym for component. (See \autoref{component-declarations}.)

\textbf{vector}: an array where the number of dimensions is 1. (See
\autoref{arrays}.)
