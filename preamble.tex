
% This suppresses warnings about most overfull hboxes.
% Clearly it would be better to fix them, but if we anyway want to
% change the layout that can wait
\hfuzz=50pt

% Uses some packages:
\usepackage{amsmath}
\allowdisplaybreaks

% For non-floating figures:
\usepackage{float}


% Prefer pdf for pdf
% https://tex.stackexchange.com/questions/1072/which-graphics-formats-can-be-included-in-documents-processed-by-latex-or-pdflat
\usepackage{ifpdf}
\ifpdf
\makeatletter
\let\orig@Gin@extensions\Gin@extensions
\def\Gin@extensions{.pdf,\orig@Gin@extensions} %prepend .pdf before .png
\makeatother
\fi

% Longer section numbers in toc
\ifpdf
\usepackage{tocloft}
\setlength{\cftsecnumwidth}{3em}
\fi

% Need more text:
\usepackage[margin=1in]{geometry}

%\usepackage{t1enc}
\usepackage{graphicx}
\graphicspath{{media/}{../media/}} % so that chapter files can also find the images

\usepackage{verbatim}
% The fixltx2e that was used for textsubscript is no longer needed for pdf, but is needed by latexml
\ifpdf
\else
\usepackage{fixltx2e} %for textsubscript
\fi

% Allow lot of figures on page:
\renewcommand{\textfraction}{0}
\renewcommand{\topfraction}{1}
\renewcommand{\bottomfraction}{1}

% Some problem with raggedright for latexml
\ifpdf
\renewcommand{\newline}{\hspace*{\fill}\linebreak}
\else
\renewcommand{\newline}{\hspace*{\fill}\raggedright\linebreak}
\fi
\setcounter{totalnumber}{1000}
\raggedbottom
\usepackage{hyperref}
% The default (magenta for urls and red for internal links) isn't nice
\hypersetup{colorlinks=true, urlcolor=blue, linkcolor=blue}
% Using longtable to have tables split across pages
% Most starting with \hline \endhead to avoid underfull vbox and not have first part duplicated on new pages
%
% But in some cases there are actual headings that are copied to next page.
% Note that this can often lead to overfull vbox
% See https://tex.stackexchange.com/questions/71096/underfull-vbox-in-each-longtable-can-it-be-fixed-or-must-it-be-ignored
\usepackage{longtable}
\usepackage{multirow} % for multirow entries in tables
\usepackage{listings}
\usepackage{color}
\usepackage[table]{xcolor}
%\def\doublelabel#1{\label{#1}\hypertarget{#1}{}}
\def\doublelabel#1{\label{#1}}
\def\S0#1#2{\hypertarget{#2}\chapter{#1}\label{#2}}
\def\S1#1#2{\hypertarget{#2}\section{#1}\label{#2}}
\def\S2#1#2{\hypertarget{#2}\subsection{#1}\label{#2}}
\def\S3#1#2{\hypertarget{#2}\subsubsection{#1}\label{#2}}
\def\S4#1#2{\hypertarget{#2}\paragraph{#1}\label{#2}}
\def\S5#1#2{\hypertarget{#2}\subparagraph{#1}\label{#2}}

% Simple numbering of lists, should preferably use \usepackage{enumitem}
% https://stackoverflow.com/questions/2007627/latex-how-can-i-create-nested-lists-which-look-this-1-1-1-1-1-1-1-2-1-2
\renewcommand{\labelenumi}{\arabic{enumi}.}
\renewcommand{\labelenumii}{\labelenumi\arabic{enumii}.}
\renewcommand{\labelenumiii}{\labelenumii\arabic{enumiii}.}
\renewcommand{\labelenumiv}{\labelenumiii\arabic{enumiv}.}

\lstset{
    basicstyle=\color{blue}\ttfamily,
    keywordstyle=\color{red}\ttfamily\bfseries,
    commentstyle=\color{green}\sffamily,
}
\setcounter{secnumdepth}{5}
% Note: Toc changed for appendex
\setcounter{tocdepth}{1}
\definecolor{keywordcolor1}{rgb}{0,0,.4}
\definecolor{keywordcolor2}{rgb}{.90,0,0}
% See https://github.com/modelica-tools/listings-modelica/blob/master/listings-modelica.cfg
% Note: Changed comment color from green to [rgb]{0,0.4,0} - since the other variant was too distracting
% And added pure,impure,stream as keywords
\lstdefinelanguage{modelica}{%
    alsoletter={...},%
    %otherkeywords={-, =, +, [, ], (, ), \{, \}, :, *, !},%
    morekeywords=[1]{},% blue Keywords
    morekeywords=[2]{% blue + bold keywords
        annotation,assert,block,break,class,connector,constant,discrete,%
        encapsulated,else,elseif,elsewhen,end,exit,extends,external,final,flow,for,%
        function,if,in,inner,initial,input,import,loop,model,nondiscrete,outer,%
        output,package,parameter,partial,record,redeclare,replaceable,return,%
        size,terminate,then,type,when,while,algorithm,equation,%
        protected,public,and,false,not,or,true,pure,impure,stream},%
    % Note: initial is in both variants - depending on context, use first variant
    morekeywords=[3]{% red keywords
        abs,acos,asin,atan,atan2,connect,cos,cosh,cross,der,edge,exp,%
        noEvent,pre,reinit,sample,sign,sin,sinh,tan,tanh,terminal,%
        start,Real,Integer,Boolean,String,homotopy,spatialDistribution,cardinality},%
    comment=[l]{//}, % comment lines
    morecomment=[s]{/*}{*/}, % comment blocs
    morestring=[b]{'}, %
    morestring=[b]{"},
}[keywords,comments,strings]
\lstdefinelanguage{grammar}{
alsoletter={...},
breaklines=true,
breakatwhitespace=true,
morekeywords=[2]{letters}
morekeywords=[1]{|}
}[keywords,comments,strings]
\lstset{ %
  backgroundcolor=\color{white},   % choose the background color
  basicstyle=\footnotesize\ttfamily,        % size of fonts used for the code
  breaklines=true,                % automatic line breaking only at whitespace
  captionpos=b,                    % sets the caption-position to bottom
  commentstyle=\color[rgb]{0,0.4,0}\sffamily,    % comment style
  keywordstyle=\color{blue}\ttfamily\bfseries,       % keyword style
  keywordstyle=[1]\color{keywordcolor1},
  keywordstyle=[2]\color{keywordcolor1}\bfseries,
  keywordstyle=[3]\color{keywordcolor2},
  stringstyle=\color{black},     % string literal style
  language=modelica,             % Set your language (you can change the language for each code-block optionally)
  showstringspaces=false,
  frame=lrtb, %
  xleftmargin=\fboxsep, %
  xrightmargin=-\fboxsep, %
}

% Duplicate this definition here to avoid issue - and name it "fortran77" to avoid problem in LatexML
\lstdefinelanguage{fortran77}%
  {morekeywords={ASSIGN,BACKSPACE,CALL,CHARACTER,%
      CLOSE,COMMON,COMPLEX,CONTINUE,DATA,DIMENSION,DO,DOUBLE,%
      ELSE,ELSEIF,END,ENDIF,ENDDO,ENTRY,EQUIVALENCE,EXTERNAL,%
      FILE,FORMAT,FUNCTION,GO,TO,GOTO,IF,IMPLICIT,%
      INQUIRE,INTEGER,INTRINSIC,LOGICAL,%
      OPEN,PARAMETER,PAUSE,PRECISION,PRINT,PROGRAM,READ,REAL,%
      RETURN,REWIND,STOP,SUBROUTINE,THEN,%
      WRITE,SAVE},%
    morekeywords=[2]{ACCESS,BLANK,BLOCK,DIRECT,EOF,ERR,EXIST,%
      FMT,FORM,FORMATTED,IOSTAT,NAMED,NEXTREC,NUMBER,OPENED,%
      REC,RECL,SEQUENTIAL,STATUS,TYPE,UNFORMATTED,UNIT},%
    morekeywords=[3]{INT,DBLE,CMPLX,ICHAR,CHAR,AINT,ANINT,% left out real
      NINT,ABS,MOD,SIGN,DIM,DPROD,MAX,MIN,AIMAG,CONJG,SQRT,EXP,LOG,%
      LOG10,SIN,COS,TAN,ASIN,ACOS,ATAN,ATAN2,SINH,COSH,TANH,LGE,LLE,LLT,%
      LEN,INDEX},%
    morekeywords=[4]{AND,EQ,EQV,FALSE,GE,GT,OR,LE,LT,NE,NEQV,NOT,TRUE},%
   sensitive=f,%% not Fortran-77 standard, but allowed in Fortran-95 %%
   morecomment=[f]*,%
   morecomment=[f]C,%
   morecomment=[f]c,%
   morestring=[d]",%% not Fortran-77 standard, but allowed in Fortran-95 %%
   morestring=[d]'%
  }[keywords,comments,strings]%
\lstdefinelanguage{C}%
  {morekeywords={auto,break,case,char,const,continue,default,do,double,%
      else,enum,extern,float,for,goto,if,int,long,register,return,%
      short,signed,sizeof,static,struct,switch,typedef,union,unsigned,%
      void,volatile,while},%
   sensitive,%
   morecomment=[s]{/*}{*/},%
   morecomment=[l]//,% nonstandard
   morestring=[b]",%
   morestring=[b]',%
   moredelim=*[directive]\#,%
   moredirectives={define,elif,else,endif,error,if,ifdef,ifndef,line,%
      include,pragma,undef,warning}%
  }[keywords,comments,strings,directives]%

% Note: \lstinline!...! used for inline Modelica
\title{Modelica® - A Unified Object-Oriented Language for Systems
Modeling\\[2\baselineskip]Language
Specification\\[2\baselineskip]Version 3.4}
\date{\today}
\author{Modelica Association}